\documentclass[12pt,a4paper]{article}
\usepackage{dsdmacros}

%\usepackage[danish]{babel} % danish may not be present, if not try "sudo apt install texlive-lang-european" a fmtutil -sys --all
%\selectlanguage{danish}

\newcommand\mymaketitle[1]{
   \rule{\textwidth}{1.6pt}\vspace*{-\baselineskip}\vspace*{2pt}
   \rule{\textwidth}{0.4pt}
   \\   
   \huge \bf #1\\
   \vspace{-8pt}
   \rule{\textwidth}{0.4pt}\vspace*{-\baselineskip}\vspace{3.2pt}
   \rule{\textwidth}{1.6pt}
}

\begin{document}

\title{
	\mymaketitle{E2DSD\\Øvelse \# - Navn på øvelse}
}
\author{
	Gruppe \#\\
	Navn1 - Studienummer\\
	Navn2 - Studienummer
}
\date{21 januar 2018}

\maketitle

\section{Øvelsestemplate}

\subsection{Introduktion}

Her besvares spørgsmålet: ”Hvorfor?”.  Baseret på din forståelse af
øvelsesvejledningen, hvad ser du som formålet med øvelsen?  3-5 linjer. 
Ligesom det indledende i en avisartikel, skal den være kort, præcist og give
læseren lyst til at læse videre.

\subsection{Design og Implementering}

Her besvares spørgsmålet: ”Hvordan?”.  Du beskriver her hvordan du har løst
opgaven.  Har du skrevet kode, indsættes den her.  Koden skal indeholde
passende kommentarer for det som ikke er trivielt eller ikke er set før. 
Indeholder din løsning hardware, beskrives denne også her.

\subsection{Resultater}

Her besvares spørgsmålet: ”Hvad?”.  Resultater i form af
simuleringwaveforms, RTL- og Technology Map Views og fotos fra tests
medtages her.  Vigtige observationer fremhæves nøgternt, f.eks.  med cirkle
og pile på figurer og med tilhørende tekst.  Du skal ikke tolke dine
resultater her, blot præsentere dem.

\subsection{Diskussion}

Dette afsnit diskuterer og tolker på resultaterne og uddyber besvarelsen af
spørgsmålet: ”Hvad?” Fremhæv og forklar sammenhænge fundet i resultaterne
(Eller ikke fundet!).  F.eks.  hvordan afspejles implementeringen i
resultaterne?  Hvordan mapper kode til RTL-Views?  Hvordan fremgår det af
simuleringen at koden implementerer det ønskede?  Hvordan illustrerer fotoet
det?  Hvad er sammenhængen imellem emne, implementering og testresultater? 
Hvis en opgave har flere underopgaver og dermed resultater, kan man, hvis
det giver mening, lægge et diskussionsafsnit ved hvert resultat i stedet for
et samlet afsnit.

\subsection{Konklusion}

Samlet, kort konlusion på øvelsen.  Hvilke mål blev opfyldt, hvilke mål kom
i ikke i mål med og hvorfor.

\section{Øvelse: Half-, full- og ripple-adders}

Dette er et eksempel på en journal lavet via templaten.

\subsection{Introduktion}

Denne øvelse undersøges implementeringen af half- / full- og ripple-carry
adders.  Formålet er at lære de forskellige kodningsformer i VHDL, samt at
arbejde med aritmetik i VHDL.  Design og Implementering I øvelse 1
implementeres en 4-bit adder.  Den kommenterede source kode er vist i snippet
herunder. Kommentarer benytter Doxygen (doxygen.org) syntax.

\lstinputlisting{Src/snippet.vhd}

%\dsdsrc{snippet.vhd}{caption text}{tbp}

\subsection{Resultater}

I øvelsen undersøges outputtet af implementeringerne: Hvad bliver koden til
rent skematisk?  Hvad er det simulerede output?  Og hvad sker der når vi
tester på DE2-boardet?

\dsdfig{rtl-view.png}{6cm}{Kredsløb programmeret med dataflow strukturen.}{tbp}

Af RTL-view’et illustreret i \figref{rtl-view.png} fremgår det at det at
half-adder koden skrevet i dataflow style syntetiseres til hhv.  en xor- og
en and- gate.

\subsection{Diskussion}

Dataflow er den mest low-level kodningsform i VHDL.  De to dataflow udtryk
fra koden i afsnit 2 oversættes direkte til to gates som det ses af
\figref{rtl-view.png}.  Det vil sige at der er tæt sammenhæng mellem det
skrevne i dataflow og implementeringen, hvilket vi også forventede.

\subsubsection{Timing simulering}

%\begin{figure}[bp]
%	\centering
%   \includegraphics[width=0.5\textwidth]{Figs/timing-simulation.png}
%	\caption{min figure}
%	\label{fig:timing-simulation}
%\end{figure}
% 
% use def/macro instead
\dsdfig{timing-simulation.png}{12cm}{Timing simulering for adder. (Bemærk at
opløsningen for figuren ikke er god nok her!)}{tbb}

I simuleringen er alle kombinationsmuligheder for A og B anvendt, og det
fremgår at outputtene følger funktionen for en full-adder.  Hvor inputtene
skifter samtidigt, forekommer der spikes (eller glitches) på outputtet, se
\figref{timing-simulation.png}.

\subsection{Diskussion}

Implementeringen af full-adderen lever funktionelt op til hensigten.  Der er
midlertidig spikes på outputtene når inputtene skifter samtidigt.  Dette sås
ikke for den funktionelle simulering og årsagen hertil skal findes i routing
delay fra FPGA pins til.... Statisk-0/1 Hazard…Karnough…

\subsubsection{Test på DE2-Board}

\dsdfig{de2board-photo.png}{8cm}{Foto af DE2-board, for test af adder.}{tbp}

Adderen blev mappet til switches og 7-segment displays, som illustreret i
\figref{de2board-photo.png}, bygget og programmeret ned på DE2-boardet. 
Ved at sætte switches, skiftede display til den korrekte sum.

\subsection{Konklusion}

Adderen fungerer funktionelt som den er designet i VHDL.  Adderen kan lave
overflow, men implementeringen tager heller ikke højde for dette.  De spikes
som sås i simuleringen, \figref{timing-simulation.png}, ses ikke i testen. 
De er for kortvarige til at kunne ses med det blotte øje på displayene.

\end{document}

