\documentclass[../journal2.tex]{subfiles}
\graphicspath{{Figs/3_2/}{../Figs/3_2/}}

\begin{document}

\subsection{Introduktion}

Denne øvelse fokuserer på brugen af konkatenerings tegnet \textit{\&}, til at manipulere arrays af bits. Her vil vi forsøge at bruge \textit{\&} til at skubbe og rotere bits.

\subsection{Design og implementering}

Som det kan ses på tabel \ref{src:shift_div} har vi initieret vores inputs som en vector fra 7 til 0, for at få 8 bit at arbejde med. De enkelte outputs for vores "Shift 1x left", "Shift 2x right" og "Rotate 3x left" er også initialiseret som en vector fra 7 til 0.
\begin{table}[!h]
    \centering
      \framebox{
        \rule{8pt}{0pt}
          \lstinputlisting[firstline=1,lastline=10]{../src/shift_div.vhd}
  }
  \caption{Entity af shift\textunderscore div}
  \label{src:shift_div}
\end{table}

Vi har brugt Dataflow stilen til at implementere vores shift og rotate arkitekture. "Shift 1x left" bruger linje 2 og 5-6, hvor der konkateneres et ekstra 0 bagerst på input a, hvilket gør denne til en vector af størelsen 9-bit. Den vektor tilføjes til signalet a\textunderscore shl\textunderscore signal, som er en vector af størelsen 9-bit. Der efter vælges de nederste 8-bits, (7 ned til 0) som sendes til output a\textunderscore shl. Det sidste bit ignoreres og udgår. Hermed er input fra a, skubbet 1 til venstre.\par
Til "Shift 2x right" bruges meget den samme ide som med "Shift 1x left". Der konkateneres bare 2 nuller foran vores input fra a, som også lægges i et signal, her a\textunderscore shl\textunderscore signal, som er en vector af størrelsen 10-bits. Istedet for at ignorere det sidste bit, ignorerer vi istedet de første 2 bits. Dermed er det kun bit 9 til 2, der bliver sendt videre som output til a\textunderscore ror. \par
For at rotere vores input, udnyttede vi det at dele vectoren op i del, og "lime" dem sammen igen med konkatenation. Vi tog altså de 3 LSB og konkatenerede dem med de resterende 5 MSB. Disse kune vi så sende direkte videre på a\textunderscore ror.

\begin{table}[H]
    \centering
      \framebox{
        \rule{8pt}{0pt}
          \lstinputlisting[firstline=12,lastline=23]{../src/shift_div.vhd}
  }
  \caption{Architecture af shift\textunderscore div}
  \label{src:shift_div2}
\end{table}


\subsection{Resultater}

For at undersøge om vores shift og rotate funktioner har fungeret, har vi kørt en functional simulering af vores system. \par

Som vi kan se på figur \ref{wvf_shl}, har inputtet rykket sig en gang mod venstre. Dette ses f.eks. af at bit nr. 5 på input, er rykket til bit nr. 6 i output. Desuden er der belvet indsat et nul på bit nr. 0 på outpur, og bit nr. 7 fra input er forsvundet.

\pic{wvf_shl}{1}{Functional waveform simulering af a\textunderscore shl}{wvf_shl}

For simuleringen af a\textunderscore shr, som kan ses på figur \ref{wvf_shr}, har vi også fået det ønskede resultat. F.eks ses det på bit 5 igen, at denne er blevet rykket 2 pladser til højre, så den nu er på 3. bits plads. Desuden er der kommet 2 nuller ind på de 2 MSB's, hvilket vi også ville forvente af en shift.

\pic{wvf_shr}{1}{Functional waveform simulering af a\textunderscore shr}{wvf_shr}

Simuleringen for a\textunderscore ror har også fået den ønskede effekt. Hvis vi kigger på bit 2, har den rykket sig 3 pladser mod højre, men da dette er længere end bit 0, har den rykket sig rundt, og er nu det 7. bit. Dette er sket for alle.

\pic{wvf_ror}{1}{Functional waveform simulering af a\textunderscore ror}{wvf_ror}

På Technology Map View er det ikke helt tydeligt hvad der sker, men vi går ud fra, da der går 2 seperate linjer til a\textunderscore ror, at dette er de 2 forskellige del af vores vector, der bliver mapped. På a\textunderscore shl kan vi se en stump der fører ud i ingenting, hvilket må betyde det er det 7. bit fra, der går ud i ingen ting. For a\textunderscore shr, er det ikke tydeligt hvad der sker

\pic{rtl_3_2}{0.5}{Technoligy Map View for shift\textunderscore div}{rtl_3_2}

For vores realisering på DE2-Boardet ser vi resultater der stemmer overens med den forventede opførsel fra vores simulering.


\pic{placeholder_image}{0.5}{Realisering på DE2}{eh}

\subsection{Diskussion}

At udnytte konkatenering (\&) til implementering af shift og rotate funktionaliteter er tydeligvis en god ide. Det er nemt, og som vi kan se på figur \ref{comp_rep_3_2}, er det ikke nødvendigt at bruge logiske elementer i vores implementering, da der blot bruges remapping af de inputs DE2 boarded får.

\pic{comp_rep_3_2}{0.6}{Compilation Report for shift\textunderscore div}{comp_rep_3_2}

\subsection{Konklusion}

Vi har udnyttet konkatenering (\&) til at implementere 2 shift logiker og 1 rotate logik. Vi har set at implemneteringen ikke kræver nogen logiske elementer for DE2 boarded.

\end{document}