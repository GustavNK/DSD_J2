\documentclass[../journal2.tex]{subfiles}
\graphicspath{{Figs/3_2/}{../Figs/3_2/}}

\begin{document}

\subsection{Introduktion}

Denne øvelse fokuserer på brugen af konkatenerings tegnet \textit{\&}, til at manipulere arrays af bits. Her vil vi forsøge at bruge \textit{\&} til at skubbe og rotere bits.

\subsection{Design og implementering}

Som det kan ses på tabel \ref{src:shift_div} har vi initieret vores inputs som en vector fra 7 til 0, for at få 8 bit at arbejde med. De enkelte outputs for vores "Shift 1x left", "Shift 2x right" og "Rotate 3x left" er også initialiseret som en vector fra 7 til 0.
\begin{table}[!h]
    \centering
      \framebox{
        \rule{8pt}{0pt}
          \lstinputlisting[firstline=1,lastline=10]{\dirsrc shift_div.vhd}
  }
  \caption{Entity af shift\textunderscore div}
  \label{src:shift_div}
\end{table}

Vi har brugt Dataflow stilen til at implementere vores shift og rotate arkitekture. "Shift 1x left" bruger linje 2 og 5-6, hvor der konkateneres et ekstra 0 bagerst på input a, hvilket gør denne til en vector af størelsen 9-bit. Den vektor tilføjes til signalet a\textunderscore shl\textunderscore signal, som er en vector af størelsen 9-bit. Der efter vælges de nederste 8-bits, (7 ned til 0) som sendes til output a\textunderscore shl. Det sidste bit ignoreres og udgår. Hermed er input fra a, skubbet 1 til venstre.\par
Til "Shift 2x right" bruges meget den samme ide som med "Shift 1x left". Der konkateneres bare 2 nuller foran vores input fra a, som også lægges i et signal, her a\textunderscore shl\textunderscore signal


\begin{table}[!h]
    \centering
      \framebox{
        \rule{8pt}{0pt}
          \lstinputlisting[firstline=12,lastline=23]{\dirsrc shift_div.vhd}
  }
  \caption{Architecture af shift\textunderscore div}
  \label{src:shift_div2}
\end{table}


\subsection{Resultater}

\subsection{Diskussion}

\subsection{Konklusion}

\end{document}