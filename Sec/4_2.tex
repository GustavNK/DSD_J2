\documentclass[../journal2.tex]{subfiles}
\graphicspath{{Figs/4_2/}{../Figs/4_2/}}

\begin{document}

\subsection{Introduktion}

I denne øvelse vil vi implementere systemet vist på figur \ref{ibd_mux}, hvor vi vil udnytte \textit{when}-stemenets. Desuden vil vi genbruge koden fra "bin2seevenseg", da vi her allerede har en fungerende binær til 7-segemtns display decoder.

\pic{ibd_mux}{0.8}{IBD af Hex MUX}{ibd_mux}

\subsection{Design og implementering}
\
Vi implementeret vores entity som på tabel \ref{src:hex_mux_ent} Først implementerer vi vores inputs og outputs igennem std\textunderscore logic\textunderscore vector. Vi bruger indsætter også en "work.all", da vi vil inkludere tidligere arbejde.

\begin{table}[H]
    \centering
      \framebox{
        \rule{8pt}{0pt}
          \lstinputlisting[firstline=1,lastline=11]{\dirsrc hex_mux.vhd}
  }
  \caption{Entity af hex\textunderscore mux}
  \label{src:hex_mux_ent}
\end{table}

Arkitecturen for vores implementeringen kan ses på tabel \ref{src:hex_mux_arch}. Først laver vi et siganl, hvor i vores outputs fra de 3 "bin2sevenseg" kan kmbineres i. Derfefter har vi lavet en række konstanter i logic vectors, som indeholder 7 segments outputs for at vise "On", "Err" og "---". Dette gør koden mere læselig senere. Vi har portmappet vores 3 \textit{bin2seveneseg} til vores inputs \textit{bin} og signalet \textit{sseg}. Her efter bruger vi endelig vores \textit{when} statements, så vi kan vælge imellem vores inputs på multiplexeresn. Her bruges \textit{sel} så til at vælge om der skal vises "On", "Err" eller outputs fra \textit{sseg} på \textit{tsseg}. Igen har vi sikret os imod uforudsete inputs, og viser derfor \textit{dash} på vores 7-segments displays, i tilfælde af dette.

\begin{table}[H]
    \centering
      \framebox{
        \rule{8pt}{0pt}
          \lstinputlisting[firstline=13,lastline=33]{\dirsrc hex_mux.vhd}
  }
  \caption{Architecture af hex\textunderscore mux}
  \label{src:hex_mux_arch}
\end{table}

\subsection{Resultater}


\subsection{Diskussion}

\subsection{Konklusion}
s
\end{document}